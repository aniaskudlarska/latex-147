% Template to use to complete Problem Set 0.
% Note that using this template is *optional*: it provides a nice foundation
% for getting started with LaTeX, but you aren't required to use it!
% If you are using ShareLaTeX, you'll want to upload this file to your account.
% Before modifying this file, we recommend trying to compile it as-is
% to see what the basic template gives.

\documentclass[12pt]{article}

\usepackage{amsmath}
\usepackage[margin=2.5cm]{geometry}
% If you want to use this package, make sure to download it from the course website!
\usepackage{csc}

% Document metadata
\title{CSC165H1 -- Problem Set 0 (SAMPLE SOLUTION)}
\author{David Liu \& Toniann Pitassi}
\date{\today}


% Document starts here
\begin{document}
\maketitle

\section*{My Courses}
% Replace this text with a list of the courses you're taking.
\begin{itemize}
\item CSC165H1, Mathematical Expression and Reasoning, D. Liu
\item CSC148H1, Introduction to Computer Science, B. Simion
\item PHL206H1, Later Medieval Philosophy, D. Black
\item USA300H1, Theories and Methods in American Studies, A. Rahr
\item WGS370H1, Utopian Visions, Activist Realities, J. Taylor
\end{itemize}


\section*{Set notation}

% Fill in the following:

\[
S_1 \cap S_2 = \{2, 4, 6, 8, 10, 12, 14\}
\]

(Note that 0 is not positive.)

\section*{A truth table}

% Look at \texttt{sample\_latex.tex} for an example of a table.

\begin{center}
\begin{tabular}{|c|c|c|c|}
\hline
$p$ & $q$ & $r$ & $(p \OR q) \IMP r$ \\
\hline
\FALSE & \FALSE & \FALSE & \TRUE  \\
\FALSE & \FALSE & \TRUE  & \TRUE  \\
\FALSE & \TRUE  & \FALSE & \FALSE \\
\FALSE & \TRUE  & \TRUE  & \TRUE  \\
\TRUE  & \FALSE & \FALSE & \FALSE \\
\TRUE  & \FALSE & \TRUE  & \TRUE  \\
\TRUE  & \TRUE  & \FALSE & \FALSE \\
\TRUE  & \TRUE  & \TRUE  & \TRUE  \\
\hline
\end{tabular}
\end{center}

(Showing a separate column for $p \OR q$ is acceptable; this is like showing an intermediate step in a calculation.)

\section*{A calculation}

% Look at \texttt{sample\_latex.tex} for an example of align*.

\begin{align*}
2^{x - 5} &= a^x \\
% TODO: fill in steps here!
\frac{2^{x - 5}}{a^x} &= 1 \\
2^{-5} \cdot \frac{2^x}{a^x} &= 1 \\
\frac{1}{32} \cdot \frac{2^x}{a^x} &= 1 \\
\left( \frac{2}{a} \right)^x &= 32 \\
x &= \log_{2/a} (32)
\end{align*}

\end{document}
